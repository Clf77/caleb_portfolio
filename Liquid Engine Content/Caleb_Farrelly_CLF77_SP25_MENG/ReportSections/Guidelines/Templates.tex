\section{Templates}
\subsection{SubSection Template}
\subsubsection{SubSubSection Template}

\subsection{Picture Template}
\begin{figure}[H]
    \centering
    \includegraphics[width=0.6\textwidth]{Images/logo_small.png}
    \captionof{figure}{\textbf{ImageCaption} \cite{einstein}}
\end{figure}




\subsection{Table Template}


\begin{longtable}[c]{| l | l | l | l |}
 
\hline
\multicolumn{1}{|c|}{\textbf{Column A}} &
\multicolumn{1}{c|}{\textbf{Column B}} &
\multicolumn{1}{c|}{\textbf{Column C}} &
\multicolumn{1}{c|}{\textbf{Column D}} \\
\hline

Things & Many Things & More Things & Even More Things \\
Things & Many Things & More Things & Even More Things \\
Things & Many Things & More Things & Even More Things \\
Things & Many Things & More Things & Even More Things \\
Things & Many Things & More Things & Even More Things \\

\hline
\endfirsthead

\caption{Table caption.} \\

\end{longtable}



\subsection{Equation Template}
\begin{equation}
    \dot\Omega_{sec}=-\frac{3\cdot n\cdot R_{body}^2 \cdot J_{2,body}^2}{2 \cdot a^2 \cdot (1-e^2)^2} \cdot \cos(I)
\end{equation}
\begin{equation}
    \dot\omega_{sec}=\frac{3}{2} \cdot J_{2,body} \cdot n \cdot \left(\frac{R_{body}}{a \cdot (1-e^2)}\right)^2\left(2-\frac{5}{2} \cdot \sin^2(I)\right)
\end{equation}
\begin{equation}
    \dot M=n-\frac{3 \cdot J_{2,body} \cdot n \cdot R^2_{body}}{4 \cdot a^2 \cdot (1-e^2)^{\frac{3}{2}}}(2-3 \cdot \sin^2(I))
\end{equation}
Where:
\begin{equation}
    n=\sqrt{\frac{\mu_{body}}{a^3}}
\end{equation}

\subsection{Inline Code}

Use the parameter after \verb|\begin{minted}| to specify your language.

\subsubsection{Java Code}
\begin{minted}{java}
// Hello.java
import javax.swing.JApplet;
import java.awt.Graphics;

public class Hello extends JApplet {
    public void paintComponent(Graphics g) {
        g.drawString("Hello, world!", 65, 95);
    }    
}
\end{minted}

\subsubsection{C code}
\begin{minted}{c}
#include <stdio.h>
int main() {
   // printf() displays the string inside quotation
   printf("Hello, World!");
   return 0;
}
\end{minted}

\subsubsection{MATLAB Code}
\begin{minted}{matlab}
function [phi, rp, D, dv] = gravityassist(mu,vinf_in)
%INPUTS
%   mu       - Gravitational parmater of the planet (G*M_p) in DU^3/TU^2
%   v_inf_in - 3x1 incoming v infinity vector in perifocal frame components
%
%OUTPUTS
%   phi - Turning angle (rad)
%   rp  -  Closest approach to planet (DU)
%   D   - Impact parameter (DU)
%   dv  - net heliocentric delta v due to flyby (DU/TU)

%find a and vinf from vis-viva equation
a = -mu / (norm(vinf_in))^2 ;

%find cos(theta) from geometry of the perifocal frame
rpUnit = [1; 0; 0]; % 
cosTheta = dot(rpUnit, -vinf_in)/norm(rpUnit)/norm(vinf_in) ;
e = -1/cosTheta ;

% find rp, phi, and D
D = sqrt( mu^2 * (e^2 - 1) / norm(vinf_in)^4 ) ;
rp = a*(1-e);
phi = 2 * asin(1/(1+rp*norm(vinf_in)^2 / mu)) ;
dv = 2 * norm(vinf_in) * sin(phi/2) ;

end
\end{minted}

\subsubsection{Brainf**k}
\begin{minted}{brainfuck}
++++++++++[>+++++++>+++++++++++>++++++++++
>+++>++++++++>+++ ++++++++<<<<<<-]>---.>+.
+++.>+.>>>--..<<++.>++.>+++.<<<--.>>>----.
<<<++.<++.--.+++++++.>>.>++.<<.----.>>>++.
\end{minted}

\subsubsection{GNU Radio Blocks}
Just include a screeshot of the GNU Radio blocks that you are using.
